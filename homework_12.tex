\documentclass[12pt]{article}

% Packages for math and formatting
\usepackage{amsmath}
\usepackage{amssymb}
\usepackage{amsthm}
\usepackage{geometry}
\usepackage{fancyhdr}
\usepackage{graphicx}
\usepackage{setspace}
\usepackage{listings}
\usepackage{xcolor}

\lstset{ %
    language=Matlab,                % 语言选择
    basicstyle=\ttfamily\small,     % 字体样式和大小
    keywordstyle=\color{blue},      % 关键字的颜色
    commentstyle=\color{gray},      % 注释的颜色
    stringstyle=\color{red},        % 字符串的颜色
    numbers=left,                   % 行号位置 (左侧)
    numberstyle=\tiny\color{gray},  % 行号的样式
    stepnumber=1,                   % 行号增量
    numbersep=5pt,                  % 行号与代码的距离
    backgroundcolor=\color{white},  % 背景色
    showspaces=false,               % 不显示空格符
    showstringspaces=false,         % 不显示字符串中的空格符
    showtabs=false,                 % 不显示制表符
    frame=single,                   % 给代码加上边框
    tabsize=4,                      % 设置制表符宽度
    captionpos=b,                   % 标题位置
    breaklines=true,                % 自动换行
    breakatwhitespace=true,         % 只在空格处换行
    title=\lstname,                 % 显示代码名称
    escapeinside={\%*}{*)},         % 在代码中加入LaTeX指令
    morekeywords={*,...}            % 自定义更多关键词
}

\setstretch{1.2}
\setlength{\parskip}{1em}

% Geometry settings for better margins
\geometry{a4paper, margin=1in}

% Header and footer
\pagestyle{fancy}
\fancyhf{}
\fancyhead[L]{Kecai Xuan}
\fancyhead[C]{Math Homework}
\fancyhead[R]{\today}
\fancyfoot[C]{\thepage}

% Custom commands for common math symbols
\newcommand{\R}{\mathbb{R}}
\newcommand{\N}{\mathbb{N}}
\newcommand{\Z}{\mathbb{Z}}

\begin{document}

\setlength{\parindent}{0pt}

\title{AMSC660 Homework \#12}
\author{Kecai Xuan}
\date{\today}
\maketitle

The code can be find at:

https://github.com/Bessgendre/minst-optimization-AMSC660-2024

\section*{Task 1}

The number of misclassified digits first rises up when PCA components increases, and then decreases. The optimal number of PCA components is around 30, which gives the lowest number of misclassified digits.

\begin{figure}[ht]
    \centering
    \includegraphics[width=0.6\textwidth]{./imgs/component.png}
\end{figure}

This is the learning curves when number of PCA components equals to 25:
\begin{figure}[ht]
    \centering
    \includegraphics[width=0.82\textwidth]{./imgs/loss.png}
\end{figure}


\section*{Task 2}

This is the learning curves when using Gauss-Newton method:
\begin{figure}[ht]
    \centering
    \includegraphics[width=0.82\textwidth]{./imgs/gauss_newton.png}
\end{figure}


\section*{Task 3}

\begin{figure}[ht]
    \centering
    \includegraphics[width=0.82\textwidth]{./imgs/SGD.png}
\end{figure}


\subsection*{Conclusions}
\begin{itemize}
    \item \textbf{Reasonable Batch Sizes:}
    \begin{itemize}
        \item Batch sizes of 64 and 128 strike the best balance between computational cost, convergence speed, and stability.
    \end{itemize}
    \item \textbf{Reasonable Step Sizes:}
    \begin{itemize}
        \item Step sizes of 0.01 are ideal for faster convergence when paired with decay strategies (e.g., 0.99).
        \item Smaller step sizes (e.g., 0.001) are also effective but slower, especially for larger batch sizes.
    \end{itemize}
    \item \textbf{Number of Epochs:}
    \begin{itemize}
        \item For batch sizes of 64 or 128, around 10,000 iterations (or approximately 150–200 epochs) are generally sufficient to reach convergence.
    \end{itemize}
    \item \textbf{Effectiveness of Decreasing Step Size:}
    \begin{itemize}
        \item Using a decreasing step size is beneficial for this problem as it stabilizes convergence, particularly for larger step sizes and smaller batch sizes.
        \item Decay strategies (e.g., 0.99) strike a good balance between speed and stability.
    \end{itemize}
\end{itemize}


\end{document}
